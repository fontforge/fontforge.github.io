\newif\ifPDF
\ifx\pdfoutput\undefined\PDFfalse
\else\ifnum\pdfoutput > 0\PDFtrue
\else\PDFtrue
\fi
\fi

\def\FontForge{\rm Font\-Forge }
\def\MF{\rm META\-font }

\documentclass[english]{ltugproc}
\usepackage[latin1]{inputenc}
\usepackage[T1]{fontenc}

\def\pfoottext{FontForge Tutorial}
\def\rfoottext{FontForge Tutorial}
\setcounter{page}{1}%

\ifPDF
\usepackage[pdftex]{color,graphicx}
\usepackage[pdftex]{hyperref}
\else
\usepackage[dvips]{graphicx}
\usepackage[dvips]{hyperref}
\fi
\usepackage{here}


\begin{document}
\title{Font Creation with \FontForge}
\author{George Williams}
%\address{444 Alan Rd.\\
%Santa Barbara, Ca.  93109, USA}
%\netaddress{gww@silcom.com}
%\personalURL{http://bibliofile.duhs.duke.edu/gww/}
\begin{abstract}
\FontForge is an open source program which allows the the creation and modification of
fonts in many standard formats. This article will start with the basic problem of converting a
picture of a letter into an outline image (used in most computer fonts).
Then I shall describe the automatic creation of
accented characters, and how to add ligatures and kerning pairs to a font, and other
advanced features.
Finally I shall
present a few tools for detecting common problems in font design.
\end{abstract}
\maketitle
\section{Introduction}
\FontForge creates fonts, allows you to edit existing fonts, and can convert
from one font format to another.
\subsection{What is a font?}
Well a hundred years ago a font was a collection of little
pieces of metal with the same height and one design each for the letters of the
alphabet and some extra symbols like punctuation.

But the world has changed. Fonts are more abstract, they are described by
data in a computer's memory. There are three main types of computer font in
use today.

\begin{figure}[ht]
 \caption{bitmap, stroked and outline fonts}
 \begin{center}
  \includegraphics[height=3.2cm]{bitmapchar}
  \includegraphics[height=3.2cm]{strokechar}
  \includegraphics[height=3.2cm]{outlinechar}
 \end{center}
\end{figure}

The simplest font type is a \emph{bitmap font} each character (actually each
glyph) in the font is a tiny little picture of that character expressed on a
rectangular grid of pixels. This format can provide the best quality font
possible with each glyph perfectly designed, but there are two main disadvantages: there needs to be a different
design for each size of the font, and these little pictures end up requiring
a large amount of memory.

The other two formats avoid these problems, but often require some reduction
in output quality.

A \emph{stroked font} expresses each glyph as a set of stems, with a line drawn down
the center of the stem, and then the line is drawn (stroked) with a pen of a
certain width.

The final type is an \emph{outline font.} Each glyph is expressed as a set
of contours, and the computer darkens the area between the contours. This format
is a compromise between the above two: it takes much less space than the bitmap
format, but more space than the stroked format, and it can provide better
looking glyphs than the stroked format but not as nice as the bitmap format.
It makes greater demands on the computer, however, as we shall see when we
discuss \hyperlink{Hints}{hints} later.

\subsection{What is a character? and a glyph?}
A character is an abstract concept, the letter ``A'' is a character, while
any particular drawing of that character is a glyph.  In many cases there is one
glyph for each character and one character for each glyph, but not always.

The glyph used for the latin letter ``A'' may also be used for the greek letter
``Alpha,'' while in arabic writing most arabic letters have at least four
different glyphs (often vastly more) depending on what other letters are around
them.

\subsection{What is a contour?}
Each glyph is composed of contours, and a contour is just a closed path. Usually
this path is composed of several curved segments called splines. Each spline is
defined by two end points and either 0, 1 or 2 control points which determine
how the spline curves. The more control points a spline has, the more flexible
it can be.

\begin{figure}[ht]
 \caption{splines with 0, 1 and 2 control points}
 \begin{center}
  \includegraphics[height=1.7cm]{splines}
 \end{center}
\end{figure}

\section{Font creation}
You may create an empty font either by invoking \FontForge with the \texttt{-new} argument on the command line
\begin{quote}
 \texttt{\$ fontforge -new}
\end{quote}
or by invoking the \texttt{New} item from the \texttt{File} menu. In either case
you should end up with a window like this:
\begin{figure}[h]
 \caption{A newly created blank font}
 \begin{center}
  \includegraphics[width=\columnwidth]{newfont}
 \end{center}
\end{figure}

Such a font will have no useful name as yet, and will be encoded with the default
encoding (usually Latin1). Use the \texttt{Element -> Font Info} menu item to correct
these deficiencies. This dialog has several tabbed sub-dialogs, the
first one allows you to set the font's various names.

\begin{figure}[!ht]
 \caption{Font name information}
 \begin{center}
  \includegraphics[width=\columnwidth]{fontinfo}
 \end{center}
\end{figure}

\begin{itemize}
 \item the family name (most fonts are part of a family of similar fonts)
 \item the font name, a name for PostScript, usually containing the family name and any style modifiers
 \item and finally a name that is meaningful to humans
\end{itemize}
If you wish to change the encoding (to \TeX{} Base or Adobe Standard perhaps) the
Encoding tab will present you with a pulldown list of known encodings. If you
are making a TrueType font then you should also go to the \texttt{General} tab and
select an em-size of 2048 (the default coordinate system for TrueType is a
little different from that of PostScript).

\section{Character creation}
Once you have done that you are ready to start editing characters; for
the sake of example, let's create a capital `C'. Double
click on the entry for ``C'' in the font view above. You should now have an
empty Outline Character window:
\begin{figure}[!ht]
 \caption{An empty character}
 \begin{center}
  \includegraphics[width=\columnwidth]{C1}
 \end{center}
\end{figure}

The outline character window contains two pal\-ettes snuggled up on the left
side of the window. The top palette contains a set of editing tools, and
the bottom palette controls which layers of the window are visible or editable.

The foreground layer contains the outline that will become part of the font.
The background layer can contain images or line drawings that help you draw
this particular character.
The guide layer contains lines that are useful on a font-wide basis (such as
a line at the x-height). Currently all layers are empty.

This window also shows the character's internal coordinate system with the
x and y axes drawn in light grey. A line representing the character's advance
width is drawn in black at the right edge of the window. \FontForge assigns a default
advance width of one em (in PostScript that will usually be 1000 units) to
the advance width of a new character.

Select the \texttt{File -> Import} menu command to import an image of the character
you are creating, assuming that you have one. It will be scaled so that it is as high as the em-square. In this
case that's too big and we must rescale the image.

\begin{figure}[ht]
 \caption{Background image}
 \begin{center}
  \includegraphics[width=.37\columnwidth]{C2}
  \includegraphics[width=.37\columnwidth]{C3} \\
 \end{center}
\end{figure}

Make the background layer editable (by selecting the \texttt{Back} checkbox
in the layers palette), move the
mouse pointer to one of the edges of the image, hold down the shift key
(to constrain the rescale to the same proportion in both dimensions),
depress and drag the corner until the image is a reasonable size. Next move
the mouse pointer onto the dark part of the image, depress the mouse and drag
the image to the correct position.

If you have downloaded the potrace or autotrace program
you can invoke \texttt{Element -> Auto\-Trace} to generate an outline
from the image (You should follow this by \texttt{Element -> Add Extrema} and \texttt{Element -> Simplify}).
But I suggest you refrain from autotracing, and trace the character yourself (results will be better).

Change
the active layer back to foreground (in the layers palette), and select
the curve point tool from the tools palette. Then move the pointer to the edge of the image
and add a point. I find that it is best to add points at places where the
curve is horizontal or vertical, at corners, or where the curve changes
inflection (A change of inflection occurs in a curve like ``S'' where the curve
changes from being open to the left to being open on the right. If you follow
these rules hinting will work better.

\begin{figure}[ht]
 \caption{Tracing 1}
 \begin{center}
  \includegraphics[]{C4}
 \end{center}
\end{figure}

It is best to enter a curve in a clockwise fashion, so the next point should
be added up at the top of the image on the flat section. Because the shape
becomes flat here, a curve point is not appropriate, rather a tangent point
is (this looks like a little triangle on the tools palette). A tangent point
makes a nice transition from curves to straight lines because the curve leaves
the point with the same slope the line had when it entered.

\begin{figure}[ht]
 \caption{Tracing 2}
 \begin{center}
  \includegraphics[]{C5}
 \end{center}
\end{figure}

At the moment this ``curve'' doesn't match the image at all, don't worry about
that we'll fix it later, and anyway it will change on its own as we continue.
Note that we now have a control point attached to the tangent point (the
little blue x). The next point needs to go where the image changes direction
abruptly. Neither a curve nor a tangent point is appropriate here, instead
we must use a corner point (one of the little squares on the tools palette).

\begin{figure}[ht]
 \caption{Tracing 3}
 \begin{center}
  \includegraphics[]{C6}
 \end{center}
\end{figure}
\pagebreak

As you see the curve now starts to follow the image a bit more closely. We continue
adding points until we are ready to close the path.

\begin{figure}[ht]
 \caption{Tracing 4}
 \begin{center}
  \includegraphics[]{C7}
 \end{center}
\end{figure}

Then we close the path just by adding a new point on top of the old start
point

\begin{figure}[ht]
 \caption{Tracing 5}
 \begin{center}
  \includegraphics[]{C8}
 \end{center}
\end{figure}

Now we must make the curve track the image more closely, to do this we must
adjust the control points (the blue ``x''es). To make all the control points
visible select the pointer tool and double-click on the curve. Then move
the control points around until the curve looks right.

\begin{figure}[ht]
 \caption{Tracing 6}
 \begin{center}
  \includegraphics[]{C9}
 \end{center}
\end{figure}
\pagebreak

Finally we set the advance width. Again with the pointer tool, move the mouse to the
width line on the right edge of the screen, depress and drag the line back
to a reasonable location.

And we are done with this character.

\section{Navigating to characters}

The font view provides one way of navigating around the characters in a font.
Simple scroll around it until you find the character you need and then double
click on it to open a window looking at that character.

Typing a character will move to that character.

But some fonts are huge (Chinese, Japanese and Korean fonts have thousands
or even tens of thousands of characters) and scrolling around the font view
is a an inefficient way of finding your character. \texttt{View->Goto}
provides a simple dialog which will allow you to move directly to
any character for which you know the name (or encoding). If your font is
a Unicode font, then this dialog will also allow you to find characters by
block name (e.g. There is a pull-down list from which you may select Hebrew
rather than Alef).

The simplest way to navigate is just to go to the next or previous glyph.
And \texttt{View->Next Glyph} and \texttt{View->Prev Glyph} will
do exactly that.

\section{Loading background images better}

In the background image of the previous example the bitmap of the letter filled the canvas of the
image (with no white borders around it). When \FontForge imported the image it
needed to be scaled once in
the program. But usually when you create the image of the letter you have
some idea of how much white space there should be around it. If your images
are exactly one em high then \FontForge will automatically scale them to be
the right size. So in the following examples all the images have exactly
the right amount of white space around them to fit perfectly in an em.

\FontForge also has the ability to import an entire bitmap font (for example a
``pk'' or ``gf'' font produced by \MF{} or the ``bdf'' format developed by
Adobe for bitmaps) to provide properly scaled
background images for all characters in a font.

\section{Creating the letter ``o'' -- consistent directions}

Let us turn our attention to the letter ``o'' which has a hole (or counter) in
the middle. Open the outline view for the letter ``o'' and import a background image
into it.

\begin{figure}[ht]
 \caption{Tracing o}
 \begin{center}
  \includegraphics[]{o1}
  \includegraphics[]{o2}
  \includegraphics[]{o3}
 \end{center}
\end{figure}

Notice that the first outline is drawn clockwise and the second
counter-clockwise. This change in drawing direction is important. Both PostScript
and TrueType require that the outer boundary of a character be drawn in a
certain direction (they happen to be opposite from each other, which is a
mild annoyance), within \FontForge all outer boundaries must be drawn clockwise,
while all inner boundaries must be drawn counter-clockwise.

If you fail to alternate directions between outer and inner boundaries you
may get results like the one on the left
\includegraphics[]{o-baddir}.
If you fail to draw the
outer contour in a clockwise fashion the errors are more subtle, but will
generally result in a less pleasing result once the character has been
rasterized.

\emph{TECHNICAL AND CONFUSING:} the exact behavior of
rasterizers varies. Early PostScript rasterizers used a ``non-zero winding
number rule'' while more recent ones use an ``even-odd'' rule. TrueType uses
the ``non-zero'' rule. The example given above is for the ``non-zero'' rule.
The ``even-odd'' rule would fill the ``o'' correctly no matter which way the
paths were drawn (though there would probably be subtle problems with
hinting).

To determine whether a pixel should be set using the even-odd rule draw a line
from that pixel to infinity (in any direction) and count the number
of contour crossings. If this number is even the pixel is not
filled. If the number is odd the pixel is filled. Using the
non-zero winding number rule the same line is drawn,
contour crossings in a clockwise direction add 1 to the crossing count,
while counter-clockwise contours subtract 1. If the result is 0 the pixel is not
filled, any other result will fill it.

The command \texttt{Element->Correct
Direction} will look at each selected contour, figure out whether
it qualifies as an outer or inner contour and will reverse the drawing direction
when the contour is drawn incorrectly.
\newline

\section{Creating letters with consistent stem widths, serifs and heights.}

Many Latin (Greek, Cyrillic) fonts have serifs, special terminators at the
end of stems. And in almost all LGC fonts there should only be a small number
of stem widths used (ie. the vertical stem of "l" and "i" should probably
be the same).

\FontForge does not have a good way to enforce consistency, but it does have
various commands to help you check for it, and to find discrepancies.

Let us start with the letter "l" and go through the familiar process of importing
a bitmap and defining it's outline.

\begin{figure}[ht]
 \caption{Beginning ``l''}
 \begin{center}
  \includegraphics[height=2.5cm]{l1}
 \end{center}
\end{figure}

Use the magnify tool to examine the bottom serif, and note that it is
symmetric left to right.

\begin{figure}[ht]
 \caption{Magnified ``l''}
 \begin{center}
  \includegraphics[]{l2}
 \end{center}
\end{figure}

Trace the outline of the right half of the serif

\begin{figure}[ht]
 \caption{Half traced ``l''}
 \begin{center}
  \includegraphics[]{l3}
 \end{center}
\end{figure}
\pagebreak

Select the outline, invoke \texttt{Edit -> Copy}, then \texttt{Edit ->
Paste}, and \texttt{Element -> Trans\-form} and select
\texttt{Flip} (from the pull down list) and \texttt{check Horizontal}

\begin{figure}[ht]
 \caption{Pasted ``l''}
 \begin{center}
  \includegraphics[]{l4}
 \end{center}
\end{figure}

Drag the flipped serif over to the left until it snuggles up against
the left edge of the character

\begin{figure}[ht]
 \caption{Dragged ``l''}
 \begin{center}
  \includegraphics[]{l5}
 \end{center}
\end{figure}

Deselect the path, and select one end point and drag it until
it is on top of the end point of the other half.

\begin{figure}[ht]
 \caption{Joining ``l''}
 \begin{center}
  \includegraphics[width=.45\columnwidth]{l6}
  \includegraphics[width=.45\columnwidth]{l7}
 \end{center}
\end{figure}
\pagebreak

Finish off the character.

\begin{figure}[!ht]
 \caption{Finished ``l''}
 \begin{center}
  \includegraphics[height=2.5cm]{l8}
 \end{center}
\end{figure}

But there are two more things we should do.
First let's measure the stem width, and
second let's mark the height of the ``l.''

Select the ruler tool from the tool palette, and drag it from one edge
of the stem to the other. A little window pops up showing the width is 58
units, the drag direction is 180 degrees, and the drag was -58 units
horizontally, and 0 units vertically.

\begin{figure}[ht]
 \caption{Measuring stem width}
 \begin{center}
  \includegraphics[]{l9}
 \end{center}
\end{figure}

Go to the layers palette and select the Guide radio box (this makes the
guide layer editable). Then draw a line at the top of the "l", this line
will be visible in all characters and marks the ascent height of this font.

\begin{figure}[ht]
 \caption{Making a guideline}
 \begin{center}
  \includegraphics[]{l10}
 \end{center}
\end{figure}
\pagebreak

The ``i'' glyph looks very much like a short ``l'' with a
dot on top. So let's copy the ``l'' into the ``i;'' this will automatically give
us the right stem width and the correct advance width. The copy may be done
either from the font view (by selecting the square with the ``l'' in it and
pressing \texttt{Edit -> Copy}) or from the outline view (by
\texttt{Edit -> Select All} and \texttt{Edit -> Copy}). Similarly
the Paste may be done either in the font view (by selecting the ``i'' square
and pressing \texttt{Edit -> Paste}) or the outline view (by opening
the ``i'' character and pressing \texttt{Edit -> Paste}).

Import the ``i'' image, and copy and paste the ``l'' glyph.

\begin{figure}[ht]
 \caption{Import ``i''}
 \begin{center}
  \includegraphics[height=2.5cm]{i1}
 \end{center}
\end{figure}

Select the top serif of the outline of the l
and drag it down to the right height

\begin{figure}[ht]
 \caption{Correcting ``i''}
 \begin{center}
  \includegraphics[width=.40\columnwidth]{i2}
  \includegraphics[width=.40\columnwidth]{i3}
 \end{center}
\end{figure}

Go to the guide layer and add a line at the x-height

\begin{figure}[ht]
 \caption{Making another guideline}
 \begin{center}
  \includegraphics[]{i4}
 \end{center}
\end{figure}

Looking briefly
back at the "o" we built before, you may notice that the "o" reaches a little
above the guide line we put in to mark the x-height (and a little below the
baseline). This is called overshoot and is an attempt to remedy an optical
illusion. A curve actually needs to rise about 3\% (of its diameter) above
the x-height for it to appear on the x-height.

\begin{figure}[ht]
 \caption{Comparing ``o'' to guidelines}
 \begin{center}
  \includegraphics[]{o5}
 \end{center}
\end{figure}

Continuing in this manner we can produce all the base glyphs of a font.


\section{Hints}
\hypertarget{Hints}{At} small point sizes on display screens, computers often have a hard time
figuring out how to convert a glyph's outline into a pleasing bitmap to display.
The font designer can help the computer out here by providing what are called ``Hints.''

Basically every horizontal and vertical stem in the font should be hinted.
\FontForge has a command \texttt{Element -> Auto\-hint} which should do this
automatically. Or you can create hints manually -- the easiest way is to select
two points on oposite sides of a stem and then invoke \texttt{Hints->Add HHint}
or \texttt{Hints->Add VHint} respectively for horizontal or
vertical stems.

\begin{figure}[ht]
 \caption{``o'' with hints}
 \begin{center}
  \includegraphics[]{o6}
 \end{center}
\end{figure}

\section{Accented letters}

Latin, Greek and Cyrillic all have a large complement of accented characters.
\FontForge provides several ways to build accented characters out of base
characters.

The most obvious mechanism is simple copy and paste:
\texttt{Copy} the letter ``A'' and
\texttt{Paste} it to ``�'' then
\texttt{Copy} the tilde accent and
\texttt{Paste it Into} ``�'' (N.B. \texttt{Paste
Into} is subtly different from \texttt{Paste}. \texttt{Paste} clears out the character before
pasting, while \texttt{Paste Into} merges the clipboard into the character, retaining the old
contents).  Then you open up ``�" and position the accent
so that it appears properly centered over the A.

This mechanism is not particularly efficient, if you change the shape of
the letter ``A'' you will need to regenerate all the accented characters built
from it. \FontForge has the concept of a
Reference to a character. So you can
\texttt{Copy a Reference} to ``A'', and \texttt{Paste} it, the \texttt{Copy a Reference} to tilde and
\texttt{Paste it Into}, and then again adjust the position of the accent over the
A.

Then if you change the shape of the A the shape of the A in ``�'' will
be updated automagically -- as will the width of ``�''.

But \FontForge knows that ``�'' is built out of ``A'' and the tilde accent,
and it can easily create your accented characters itself by placing the
references in ``�'' and then positioning the accent over the ``A''. (Unicode
provides a database which lists the components of every accented character
(in Unicode)). Select ``�,'' then apply \texttt{Element -> Build -> Build Accented}
and \FontForge will create the character by pasting references to the two components
and positioning them appropriately.

\FontForge has a heuristic for positioning accents -- most accents are centered
over the highest point of the character -- sometimes this will produce bad
results (if the one of the two stems of ``u'' is slightly taller than the other
the accent will be placed over it rather than being centered over the character),
so you should be prepared to look at your accented characters after creating
them. You may need to adjust one or two (or you may need to redesign your
base characters slightly).

\section{Ligatures}

One of the great drawbacks of the standard Type1 fonts from Adobe is that none of
them comes with ``ff'' ligatures. Lovers of fine typography tend to object to this.
\FontForge can help you overcome this flaw (whether it is legal to do so is a
matter you must settle by reading the license agreement for your font).
\FontForge cannot create a nice ligature for you, but what it can do is put
all the components of the ligature into the character with
\texttt{Element -> Build -> Build Composite}.
This makes it slightly easier (at least in latin) to design a ligature.

Use the \texttt{Element -> Char Info} dialog to name the character (in this case to
``ffi''. This is a standard name and \FontForge recognizes it as a ligature
consisting of f, f and i). Apply \texttt{Element -> Default ATT -> Common Ligatures}
so that \FontForge will store the fact that it is a ligature. Then use
\texttt{Element -> Build -> Build
Composite} to insert references to the ligature components.

\begin{figure}[ht]
 \caption{ffi made of references}
 \begin{center}
  \includegraphics[]{ffi-refs}
 \end{center}
\end{figure}

Use \texttt{Edit -> Unlink References}
to turn the references into a set of contours.

\begin{figure}[ht]
 \caption{ffi without references}
 \begin{center}
  \includegraphics[]{ffi-unlink}
 \end{center}
\end{figure}

Adjust the components so that they will look better together. Here the
stem of the first f has been lowered.

\begin{figure}[ht]
 \caption{ffi adjusted}
 \begin{center}
  \includegraphics[]{ffi-moved}
 \end{center}
\end{figure}

Use \texttt{Element -> Remove Overlap}
to clean up the character.

\begin{figure}[!ht]
 \caption{ffi cleaned up}
 \begin{center}
  \includegraphics[]{ffi-rmoverlap}
 \end{center}
\end{figure}

Some word processors will allow the text editing caret to be placed inside a ligature
(with a caret position between each component of the ligature). This means
that the user of that word processor does not need to know s/he is dealing
with a ligature and sees behavior very similar to what s/he would see if
the components were present. But if the word processor is to be able to do
this it must have some information from the font designer giving the appropriate locations
of caret positions. As soon as \FontForge notices that a character
is a ligature it will insert enough caret location lines into it to fit between
the ligature's components. \FontForge places these on the origin, if you leave
them there \FontForge will ignore them. But once you have built your
ligature you might want to move the pointer tool over to the origin line,
press the button and move the caret lines to their correct locations.
(Only AAT and OpenType support this).

\begin{figure}[ht]
 \caption{ffi with ligature carets}
 \begin{center}
  \includegraphics[]{ffi-caret}
 \end{center}
\end{figure}

\section{Metrics}
Once you have created all your glyphs, you should probably examine them to see
how they look together.  There are three commands designed for this:
\begin{itemize}
 \item \texttt{Windows -> New Metrics View} -- will open a window which
displays several glyphs at a very large size. You can change the advance
width of each glyph here to make a more pleasing image.
 \item \texttt{File -> Print} -- will print a sample text using the font, or
all the glyphs of the font, or several glyphs one per page, or several glyphs
at a waterfall of point sizes.
 \item \texttt{File -> Display} -- will open a dialog which allows you
to display a sample text in this (or indeed several) fonts.
\end{itemize}

\section{Kerning}

Even in fonts with the most carefully designed metrics there are liable to be
some character combinations which look ugly. Some combinations are fixed
by building ligatures, but most are best approached by kerning the inter-character
spacing for that particular pair.

\begin{figure}[ht]
 \caption{kerning in the Metrics view}
 \begin{center}
  \includegraphics[width=.43\columnwidth]{To-unkerned}
  \includegraphics[width=.43\columnwidth]{To-kerned}
 \end{center}
\end{figure}

In the above example the left image shows the unkerned text, the right shows
the kerned text. To create a kerned pair, select the two glyphs, then use
\texttt{Windows -> New Metrics View} and move the mouse to the rightmost character
of the pair and click on it, the line (normally the horizontal advance) between the two should
go green (and becomes the kerned advance). Drag this line around until the
spacing looks nice.

\section{Checking a font for common problems}

After you have finished making all the characters in your font you should
check it for inconsistencies. \FontForge has a command,
\texttt{Element -> Find Problems} which is designed to
find many common problems (as you might guess).

Simply select all the characters in the font and then bring up the Find Problems
dialog. 
Be warned though: Not everything it reports as a problem is a real problem,
some may be an element of the font's design that \FontForge does not expect.

The dialog can search for many types of problems:
\begin{itemize}
 \item Stems which are close to but not exactly some standard value
 \item Points which are close to but not exactly some standard height
 \item Paths which are almost but not quite vertical or horizontal
 \item Control points which are in unlikely places
 \item Points which are almost but not quite on a hint
 \item and more.
\end{itemize}
I find it best just to check for a few similar problems at a time,
otherwise switching between different types of problems can be distracting.

\section{Generating a font}

The penultimate\footnote{
The final stage of font creation would be installing the font. This depends
on what type of computer you use and I shan't attempt to describe all the
possibilities here.}
stage of font creation is generating a font. N.B.: \FontForge's
\texttt{File -> Save} command will produce a format that is only understood
by \FontForge and is not useful in the real world.

You should use \texttt{File -> Generate} to convert your font
into one of the standard font formats. \FontForge presents what looks like a
vast array of font formats, but in reality there are just several variants
on a few basic font formats: PostScript Type 1, TrueType, OpenType (and for
CJK fonts, also CID-keyed fonts) and SVG.

\section{OpenType advanced typography}
In OpenType and Apple's Advanced Typography fonts it is possible for the font
to know about certain common glyph transformations and provide information
about these to a word processor using that font (which presumably could then
allow the user access to that transformation).

\subsection{Simple substitutions}
Suppose that we had a font with several sets of digits: monospaced digits,
proportional digits and lower case (old style) digits. One of these styles
would be chosen to represent the digits by default (say the monospaced digits).
Then we could link the default glyphs to their variant forms.

First we should name each glyph appropriately (proportional digits should be
named ``zero.fitted,'' ``one.fitted'' and so forth, while oldstyle digits should be
named ``zero.oldstyle,'' ``one.oldstyle'' and so forth. Use the
\texttt{Element-> Glyph Info} command to name them.

\begin{figure}[ht]
 \caption{Naming a glyph}
 \begin{center}
  \includegraphics[height=5cm]{glyphinfo-name}
 \end{center}
\end{figure}
\pagebreak

To link the glyphs together we invoke \texttt{Element -> Glyph Info} again,
this time on the default glyph and select the \texttt{Subs} tab. This provides
a list of all simple substitutions defined for this glyph.

\begin{figure}[ht]
 \caption{Providing substitutions}
 \begin{center}
  \includegraphics[height=5cm]{glyphinfo}
 \end{center}
\end{figure}

Pressing the \texttt{[New]} button will allow you to add a substitution

\begin{figure}[ht]
 \caption{Editing substitutions}
 \begin{center}
  \includegraphics[height=3.5cm]{taglang}
 \end{center}
\end{figure}
\pagebreak

Each substitution must contain: the name of a glyph to which it is to be mapped,
a four character OpenType tag used to identify this mapping and a script and
language in which this substitution is active. There is a pulldown menu which
you can use to find standard tags for some common substitutions (the tag for
oldstyle digits is `onum'). This substitution is for use in the latin script
and for any language, again there is a pulldown menu to help chose this
correctly.

\subsection{Contextual substitutions}
OpenType and Apple also provide contextual substitutions. These are substitutions
which only take place in a given context and are essential for typesetting
Indic and Arabic scripts.

In OpenType a context is specified by a set of patterns that are tested against
the glyph stream of a document. If a pattern matches then any substitutions it defines will
be applied.

Instead of an Indic example, let us take something I'm more familiar with,
the problem of typesetting a latin script font
where the letters ``b,'' ``o,'' ``v'' and ``w'' join their following letter
near the x-height, while all other letters join near the baseline.

Thus we need two variants for each glyph, one that joins (on the left) at the
baseline (the default variant) and one which joins at the x-height. Let us
call this second set of letters the ``high'' letters and name them ``a.high,''
``b.high'' and so forth.

\begin{figure}[ht]
 \caption{Incorrect \& correct script joins}
 \begin{center}
  \includegraphics[height=2cm]{bed-script}
 \end{center}
\end{figure}

We divide the set of possible glyphs into three classes: the letters ``bovw'',
all other letters, and all other glyphs. We need to create two patterns, the
first will match a glyph in the ``bovw'' class followed by a glyph in the ``bovw''
class, while the second will match a glyph in the ``bovw'' class followed by
any other letter. If either of these matches the second glyph should be
transformed into its high variant.

The first thing we must do is create a simple substitution mapping each low letter
to its high variant. Let us call this substitution by the four character OpenType
tag ``high.'' We use \texttt{Element -> Glyph Info} as before
except that here we use the special ``script / language'' called ``--- Nested ---''
(an option in the pulldown menu).

The tricky part is defining the context. This is done with the \texttt{Contextual}
tab in the \texttt{Element -> Font Info} dialog, revealing five different
types of contextual behavior, we are interested in contextual chaining
substitutions.

\begin{figure}[ht]
 \caption{Font Info showing Contextual Subs}
 \begin{center}
  \includegraphics[height=6cm]{ott1}
 \end{center}
\end{figure}

You can add a new entry by pressing the \texttt{[New]} button. This brings up
a series of dialogs, the first requests a four character OpenType tag and a
script / language (much as we saw earlier).
The next dialog allows you to specify the overall format of the substitution.

\begin{figure}[ht]
 \caption{Tag \& Script dialog and format of contextual chaining substitution}
 \begin{center}
  \includegraphics[height=3.3cm]{ott2}
  \includegraphics[height=4cm]{ott2_5}
 \end{center}
\end{figure}

The next dialog finally shows something interesting. At the top are a
series of patterns to match and substitutions that will be applied if the
string matches. Underneath that are the glyph classes that this substitution
uses.

A contextual chaining dialog divides the glyph stream into three categories:
those glyphs before the current glyph (these are called backtracking glyphs),
the current glyph itself (you may specify more than one), and this (these)
glyphs may have simple substitutions applied to them, and finally glyphs
after the current glyph (these are called lookahead glyphs).

Each category of glyphs may divide glyphs into a different set of classes, but
in this example we use the same classes for all categories (this makes it easier
to convert the substitution to Apple's format).

The first line (in the ``List of lists'' field) should be read thus: If a
backtracking glyph in class 1 is followed by a match glyph in class 2, then
location 0 in the match string (that is the first glyph) should have simple
substitution `high' applied to it. If you look at the glyph class definitions
you will see that class 1 includes those glyphs which must be followed by a
high variant, so this seems reasonable.

The second line is similar except that it matches glyphs in class 1. Looking
at the class definitions we see that classes 1 \& 2 include all the letters,
so these two lines mean that if any letter follows one of ``bovw'' then that
letter should be converted to its `high' variant.

\begin{figure}[ht]
 \caption{Overview of the contextual chaining substitution}
 \begin{center}
  \includegraphics[height=4.8cm]{ott3}
 \end{center}
\end{figure}

To edit a glyph class simply double click on it. To create a new one press
the \texttt{[New]} button (under the class list).

\begin{figure}[ht]
 \caption{Editing glyph classes}
 \begin{center}
  \includegraphics[height=4.8cm]{ott4}
 \end{center}
\end{figure}
\pagebreak

This produces another dialog showing all the names of all the glyphs in the current
class. Pressing the \texttt{[Select]} button will set the selection in the font window
to match the glyphs in the class, while the \texttt{[Set]} button will do the
reverse and set the class to the selection in the font window. These provide
a short cut to typing in a lot of glyph names. Pressing the \texttt{[Next]} button
defines the class and returns to the overview dialog.

To edit a pattern double click on it, or to create a new one press the 
\texttt{[New]} button (under the List of lists).

\begin{figure}[ht]
 \caption{Adding matches and substitutions}
 \begin{center}
  \includegraphics[height=5cm]{ott5}
  \includegraphics[height=2.3cm]{ott6}
 \end{center}
\end{figure}

Again the pattern string is divided into three categories, those glyphs before the current one,
the current one itself, and any glyphs after the current one. You choose which
category of the pattern you are editing with the tabs at the top of the dialog.
Underneath these is the subset of the pattern that falls within the current
category, the classes defined for this category, and finally the substitutions
for the current glyph(s). Clicking on one of the classes will add the class
number to the pattern.

To edit a substitution double click on it, or to create a new one press the 
\texttt{[New]} button (under ``An ordered list...''). The sequence number
specifies which glyph among the current glyphs should be modified, and the
tag specifies a four character substitution name

\section{Apple advanced typography}
Some of Apple's typographic features can be readily interconverted into
equivalent OpenType features, while others cannot be.

Non-contextual ligatures, kerning and substitutions can generally be converted
from one format to another. Apple uses a different naming convention and defines
a different set of features, but as long as a feature of these types is named
in both systems interconversion is possible.

\subsection{Contextual substitutions}
Apple specifies a context with a finite state machine, which is essentially
a tiny program that looks at the glyph stream and decides what substitutions
to apply.

Each state machine has a set of glyph class definitions (just as in the
OpenType example), and a set of states. The process begins in state 0 at the
start of the glyph stream. The computer determines what class the current glyph
is in and then looks at the current state to see how it will behave when
given input from that class. The behavior includes the ability to change to
a different state, advancing the input to the next glyph, applying a substitution
to either the current glyph or a previous one (the ``marked'' glyph).

Using the same example of a latin script font...

We again need a simple substitution to convert each letter into its high
alternate. The process is the same as it was for OpenType, and indeed we
can use the same substitution.

Again we divide the glyphs into three classes (Apple gives us some extra
classes whether we want them or no, but conceptually we use the same three
classes as in the OpenType example). We want a state machine with two states
(again Apple gives us an extra state for free, but we shall ignore that),
one is the start state (the base state -- where nothing changes), and the
other is the state where we've just read a glyph from the ``bovw'' class.

\begin{figure}[ht]
 \caption{State machine}
 \begin{center}
  \includegraphics[width=\columnwidth]{sm-picture}
 \end{center}
\end{figure}

Again we use the \texttt{Element -> Font Info} dialog and the \texttt{Mac SM}
tag to look at the contextual substitutions available. Again there are several
types of contextual behavior, and we are interested in contextual substitutions.

\begin{figure}[ht]
 \caption{Font Info showing State Machines}
 \begin{center}
  \includegraphics[height=6cm]{asm1}
 \end{center}
\end{figure}
\pagebreak

Double clicking on a state machine, or pressing the \texttt{[New]} button
provides an overview of the given state machine. At the top of the dialog
we see a field specifying the feature / setting of the machine, this is
Apple's equivalent of the OpenType 4 character tag. Under this is a set of
class definitions, and at the bottom is a representation of the state
machine itself.

\begin{figure}[ht]
 \caption{Overview of a State Machine}
 \begin{center}
  \includegraphics[height=6.5cm]{asm2}
 \end{center}
\end{figure}

Double clicking on a class brings up a dialog similar to that used in
OpenType:

\begin{figure}[ht]
 \caption{Editing Apple glyph classes}
 \begin{center}
  \includegraphics[height=3cm]{asm3}
 \end{center}
\end{figure}
\pagebreak

Clicking on a transition in the state machine (there is a transition for each
state / class combination) produces a transition dialog.

This controls how the state machine behaves when it is in a given state and
receives a glyph in a given class. In this example it is in state 2 (which means
it has already read a ``bovw'' glyph), and it has received a glyph in class 4
(which is another ``bovw'' glyph). In this case the next state will be state 2
again (we will have just read a new ``bovw'' glyph), read another glyph and
apply the ``high'' substitution to the current glyph.

At the bottom of the dialog are a series of buttons that allow you
to navigate through the transitions of the state machine.

\begin{figure}[ht]
 \caption{Transition dialog}
 \begin{center}
  \includegraphics[height=5cm]{asm4}
 \end{center}
\end{figure}

Pressing \texttt{[OK]} many times will extract you from this chain of
dialogs and add a new state machine to your font.

\vspace*\fill

\pagebreak	%start appendix in new column...

\section{Appendix: Additional features}

\FontForge provides many more features, further de\-scrip\-tions of which may be found at\\
\hspace*{1cm}\url{http://fontforge.sf.net/overview.html}\\
Here is a list of some of the more
useful of them:
\begin{itemize}
 \item Users may edit characters composed of either third order B�zier splines
(for PostScript fonts) or second order B�ziers (for TrueType fonts) and may
convert from one format to another.
 \item \FontForge will retain both PostScript and TrueType hints, and can
automatically hint Post\-Script fonts.
 \item \FontForge allows you to modify most features of OpenType's GSUB, GPOS and GDEF
tables, and most features of Apple's morx, kern, lcar and prop tables. Moreover
it can often convert from one format to another.
 \item \FontForge has support for Apple's font formats. It can read and generate
Apple font files both on and off a Macintosh. It can generate the FOND resource
needed for the Mac to place a set of fonts together as one family.
 \item \FontForge allows you to manipulate strikes of bitmap fonts as well as outline fonts.
It has support for many formats of bitmap fonts (including TrueType's embedded
bitmap -- both the format prescribed by Apple and that specified by MicroSoft).
 \item \FontForge can interpolate between two fonts (subject to certain constraints)
to yield a third font
between the two (or even beyond). For instance given a ``Regular'' and a ``Bold''
variant it could produce a ``DemiBold'' or even a ``Black'' variant.
 \item \FontForge also has a command (which often fails miserably) which attempts
to change the weight of a font.
 \item \FontForge can automatically guess at widths for characters, and even
produce kerning pairs automatically.
 \item \FontForge has some support for fonts with vertical metrics (in Japanese,
Chinese and Korean fonts), and some support for right to left fonts (Arabic,
Hebrew, Cypriot, etc.).
 \item \FontForge has some support for PostScript (and pdf) type3 fonts. Allowing
for glyphs with different strokes and fill and images.
 \item \FontForge has a scripting language which allows batch processing of many
fonts at once.
\end{itemize}

\end{document}
